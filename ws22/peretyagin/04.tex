\documentclass[12pt]{article}

\usepackage[utf8]{inputenc}
\usepackage[russian]{babel}

\usepackage{amsmath}
\usepackage{amsfonts}
\usepackage{amssymb}
\usepackage{wasysym}


\begin{document}

\section{Простые формулы}

Конек \TeX-а~--- это формулы, как встраиваемые
типа $|\sin x| \leqslant 1$, так и выключенные вроде

$$
    e^x = \sum_{n = 0}^\infty \frac{x^n}{n!}
        = 1 + \sum_{n = 1}^\infty \frac{x^n}{n!}.
$$


\[
    e^x = \sum_{n = 0}^\infty \frac{x^n}{n!}
        = 1 + \sum_{n = 1}^\infty \frac{x^n}{n!}.
\]



Команда \textbackslash frac хитроумная:
она выбирает размер шрифта в зависимости от того,
является формула строчной ($\frac{a}{b}$) или выключенной:
$$\frac{a}{b}.$$ Команда \textbackslash dfrac сохраняет размер:
$\dfrac{a}{b}$. Впрочем, это может показаться неэстетичным,
ведь межстрочный интервал меняется по ходу абзаца!
(Это предложение написано специально, чтобы абзац был достаточно длинным
и было можно визуально оценить изменение межстрочного интервала.)

Стрелки и прочая диакритика никогда не была проблемой
\TeX-а: $\vec a \times \vec b$, $\hat c \cdot \tilde d$,
$\grave e$, $\ddot f$, $\bar g$, $\check \imath$, $\acute \jmath$,
$\dot k$, $\breve l$.

\end{document}
