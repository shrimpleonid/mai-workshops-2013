\documentclass[12pt]{article}

\usepackage[utf8]{inputenc}
\usepackage[russian]{babel}
\usepackage{textcomp}

\begin{document}

\section{Размеры шрифтов}

\begin{tabular}{|l|l|}
\hline
\bfseries Команда & \bfseries Образец \\
\hline
\textbackslash tiny & \tiny\strut Крошечный \\
\textbackslash scriptsize & \scriptsize\strut Очень маленький (как в индексах) \\
\textbackslash footnotesize & \footnotesize\strut Маленький (как в сносках) \\
\textbackslash small & \small\strut Уменьшенный \\
\textbackslash normalsize & \normalsize\strut обычный \\
\textbackslash large & \large\strut Увеличенный \\
\textbackslash Large & \Large\strut Большой \\
\textbackslash LARGE & \LARGE\strut Очень большой \\
\textbackslash huge & \huge\strut Громадный \\
\textbackslash Huge & \Huge\strut Гигантский \\
\hline
\end{tabular}


\section{Начертания}

\begin{tabular}{|r|p{5cm}|l|}
\hline
\bfseries Без аргументов & \bfseries С аргументом & \bfseries Образец \\
\hline
\textbackslash rmfamily Обычный & \textbackslash textrm\{Обычный\} & \textrm{Обычный} \\
\textbackslash sffamily Рубленый & \textbackslash textsf\{Рубленый\} & \textsf{Рубленый} \\
\textbackslash ttfamily Моноширинный & \textbackslash texttt\{Моноширинный\} & \texttt{Моноширинный} \\
\textbackslash mdseries Обычный & \textbackslash texttt\{Обычный\} & \textmd{Обычный} \\
\textbackslash bfseries Полужирный & \textbackslash texttt\{Полужирный\} & \textbf{Полужирный} \\
\textbackslash upshape Прямой & \textbackslash textup\{Прямой\} & \textup{Прямой} \\
\textbackslash itshape Курсив & \textbackslash textit\{Курсив\} & \textit{Курсив} \\
\textbackslash slshape Наклонный & \textbackslash textsl\{Наклонный Наклонный Наклонный Наклонный Наклонный Наклонный Наклонный\} & \textsl{Наклонный} \\
\textbackslash scshape Капитель & \textbackslash textsc\{Капитель\} & \textsc{Капитель} \\
\hline
\end{tabular}


\section{Списки}

\subsection{Ненумерованные}

Любой школьник может не~знать про индукцию и~иметь двойку по физике,
но про следующие изобретения Теслы он вам точно расскажет все:
\begin{itemize}
    \item Тесла-реактор;
    \item Тесла-боец;
    \item Кольцо Тесла;
    \item Тесла-танк;
    \item Броня Тесла;
    \item Тесла-корабль\footnote{Имеется ввиду корабль с~пушкой Теслы?};
    \item Башня Тесла;
    \item Пушка Теслы.
\end{itemize}

\subsection{Нумерованные}

Бизнес-план гномов:
\begin{enumerate}
    \item Collect underpants (собрать трусы)
    \item ???
    \item PROFIT
\end{enumerate}

\begin{itemize}
    \item[sdsd:] Тесла-реактор;
    \item[sdsd:] Тесла-боец;
    \item[sdsd:] Кольцо Тесла;
    \item[sdsd:] Тесла-танк;
    \item[sdsd:] Броня Тесла;
    \item[sdsd:] Тесла-корабль\footnote{Имеется ввиду корабль с~пушкой Теслы?};
    \item[sdsd:] Башня Тесла;
    \item[sdsd:] Пушка Теслы.
\end{itemize}

\end{document}
