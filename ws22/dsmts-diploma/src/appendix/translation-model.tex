
\subsection{Модель перевода}

\frame{
	\frametitle{Введем обозначения}
		\begin{itemize}
			\item $\TE$ --- <<английский>> текст (множество предложений);
			\item $\TR$ --- <<русский>> текст;
			\item $\SE$ --- <<английское>> предложение (последовательность слов);
			\item $\SR$ --- <<русское>> предложение;
			\item $\WE$ --- <<английское>> слово;
			\item $\WR$ --- <<русское>> слово;
			%\item $\LE 	\gets |\SE|	$; 
			%\item $\LR 	\gets |\SR|	$;
			%\item $\POSR 	\gets \text{ позиция } \WR \text{ в } \SR$; 
			%\item $\POSE 	\gets \text{ позиция } \WE \text{ в } \SE$.
		\end{itemize}	
}
	
\frame[allowframebreaks]{\frametitle{Модель перевода}
	{ \small
		Пусть $P(\SE|\SR)$ --- вероятность некоторой строки (предложения) из $e$, при гипотезе перевода из $r$.
		\[
			P(\SE|\SR) = \sum\limits_{\A} P(\SE, \A|\SR);
		\]
		$\A$ --- выравнивание между отдельными словами в паре предложений.
		Вероятность перевода:
		\[
			P(\SE, \A| \SR)	= \dfrac{\varepsilon}{(\LR + 1)^{\LE}} \prod\limits_{j = 1}^{\LE} t(\WE_{j} | \WR_{\A(j)}) 
		\]
		$t$ --- это вероятность слова оригинала в позиции $j$ при соответствующем 
		ему слове перевода $\WR_{\A(j)}$, определенном выравниванием $\A$. 
	}
	\pagebreak
	{ \small
		\[
			P(\A | \SE, \SR) = \dfrac{P(\SE, \A| \SR)}
				{\sum\limits_{\A} P(\SE, \A| \SR)}
		\]
		Имея набор выравниваний с определенными вероятностями, 
		мы можем подсчитать частоты каждой пары слов, 
		\[
			t(\WE|\WR) 
				=  \dfrac{counts(\WE|\WR)}{\sum\limits_{\WE}counts(\WE|\WR)} 
				= \dfrac{counts(\WE|\WR)}{total(\WR)};
		\]
		Требуется оценить вероятности \textit{лексического перевода} $t(\WE|\WR)$ 
		Но чтобы сделать это нужно вычислить $\A$, которой у нас нет.
	}
	
	\pagebreak
	
	Для оценки параметров $\longrightarrow$ EM-алгоритм (Витерби).
	\begin{itemize}
		\item инициализируем параметры модели (одинаковыми значениями, на первой итерации);
		\item оценим вероятности отсутствующей информации;
		\item оценим параметры модели на основании новой информации;
		\item перейдем к следующей итерации.
	\end{itemize}
}

\frame{
	\scriptsize
	\begin{codebox}
		\Procname{Базовый-алгоритм($\TE$, $\TR$)}
		\li 	$\forall \ \WE \in \SE \in \TE $ \Do 	
		\li 		$\forall \WR \in \SR \in \TR $ \Do 
		\li 			$ t(\WE|\WR) \gets u, u \in \mathbb{R}$;
					\End	
				\End	
		\li 	\Com{ Инициализируем таблицу $t(\WE|\WR)$ одинаковыми значениями.}
		\li 	\Пока {не сойдется} \Do 
		\li 		$\forall \ \WE \in \SE \in \TE $ \Do \Com{ Инициализируем остальные таблицы.}
		\li 			$\forall \WR \in \SR \in \TR $ \Do 
		\li 				$ counts(\WE|\WR) \gets  0$;  $\quad total(\WR) \gets  0$;
						\End	
					\End	
		\li 		$\forall \ \SE, \SR \in \TE, \TR$ \Do \Com{ Вычисляем нормализациию. }
		\li 			$\forall \ \WE \in \SE$ \Do
		\li 				$stotal(\WE) \gets  0$;
		\li 				$\forall \ \WR \in \SR$ \Do	 
		\li 					$stotal(\WE) \gets stotal(\WE) + t(\WE|\WR)$;
		 					\End	
		 				\End	
		\li 			$\forall \ \WE \in \SE$ \Do \Com{ Собираем подсчеты. }
		\li 				$\forall \ \WR \in \SR$ \Do	 
		\li 					$counts(\WE|\WR) \gets counts(\WE|\WR) + \dfrac{t(\WE|\WR)}{stotal(\WE)}$;
		\li 					$total(\WR) \gets total(\WR) + \dfrac{t(\WE|\WR)}{stotal(\WE)}$;
			 				\End	
						\End
					\End
		\li 		$\forall \ \WE \in  \TE $ \Do \Com{ Оцениваем вероятность.}
		\li 			$\forall \WR \in \TR $ \Do 
		\li 				$ t(\WE|\WR) \gets \dfrac{counts(\WE|\WR)}{total(\WR)}$;
						\End	
					\End	
				\End
	\end{codebox}
}

